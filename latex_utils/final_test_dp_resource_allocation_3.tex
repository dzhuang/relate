\BLOCK{- if show_question -}
某公司打算向其3个销售区域增设一共\VAR{dp.total_resource}个营销点。在各销售区域增设不同数量营销点有不同的利润回报,如下表\ref{dp}所示。问:如何增设可使总利润最大?最大总利润是多少?建立动态规划模型并求解。表中被划去的单元格表示不允许相应的方案。

\begin{table}[!h]
 \centering
 \topcaption{各销售区域营销点增加个数与利润回报~~~单位:万元}\label{dp}
 \begin{mytabular}{c|\BLOCK{for d in dp.project_list}c\BLOCK{if not loop.last}|\BLOCK{endif}\BLOCK{endfor}}
 \hline\hline
\diagbox{营销点增加数}{收益$g_k$}{销售区域}&\VAR{dp.project_list|join("&")}\\ \hline
\BLOCK{for idx in range(dp.decision_set|length)}
\VAR{dp.decision_set[idx]}&\VAR{dp.gain_str_list_transposed[idx]|join("&")}\\ \hline
\BLOCK{endfor}
 \hline
 \end{mytabular}
 \end{table}

\BLOCK{endif} \#{show_question}

\key{\daan{

\BLOCK{if after_question_description}\VAR{after_question_description}\BLOCK{endif}

\BLOCK{if show_answer_explanation}
{\hei 1、动态规划建模\red{(7分,未建模扣7分)}}

(1) 阶段数$n=\VAR{dp.n_stages}$,以\VAR{dp.project_list|join("、")}作为次序,以第$k$个区域的决策作为第$k$阶段。$k=\VAR{range(1, dp.n_stages+1)|join(",")}$\textcolor[rgb]{1.00,0.00,0.00}{(1分)};

(2) 状态变量$s_k$为第$k$阶段初**尚未增设的销售点数**,则$s_1=\VAR{dp.total_resource}$(亦即$S_1=\{\VAR{dp.total_resource}\}$);

(3) 决策变量$x_k$为第$k$阶段向第$k$个区域增设的销售点数,$k=\VAR{range(1, dp.n_stages+1)|join(",")}$;

(4) 状态转移方程为:
\begin{align*}
 &s_{k+1}=s_k-x_k, ~~k=\VAR{range(1, dp.n_stages+1)|join(",")}
\end{align*}
其中$s_1=\VAR{dp.total_resource}$;

(5) 动态规划基本方程为:
\VAR{dp.get_formula_tex()}
其中,阶段指标$g_k(s_k, x_k)$为第$k$阶段向第$k$个区域增设$x_k$个销售点带来的收益。
\BLOCK{endif}\#{show_answer_explanation}
