\BLOCK{- if show_question -}
设某种物资有\VAR{dp.supply|length}个产地和\VAR{dp.demand|length}个销地:产地$A_1$,$A_2$各产\VAR{dp.supply[0]}与\VAR{dp.supply[1]}个单位;销地\BLOCK{for idx in range(1, dp.demand|length + 1)}$B_{\VAR{idx}}$\BLOCK{if not loop.last},\BLOCK{endif}\BLOCK{endfor}各销\VAR{dp.demand|join(",")}个单位。从产地运往销地,除单位运费外还有一定的固定费用。
例如,从$A_1$到$B_2$的运费由两部分构成:固定费用为\VAR{dp.cost[0][1][0]};单位运费为\VAR{dp.cost[0][1][1]}。这样,当运量$x_{12}>0$时,运费按$\VAR{dp.cost[0][1][0]}+\VAR{dp.cost[0][1][1]}\times x_{12}$计算,当运量$x_{12}=0$时,运费是0。在这里引入符号$*$来表示这种关系,如下表。


\BLOCK{ set cost=dp.cost }
<p align="middle"></br>

\begin{table}[H]\centering
\topcaption{}
\begin{mytabular}{c|\BLOCK{for idx in range(1, dp.demand|length + 1)}c|\BLOCK{endfor}c}\hline\hline
 \diagbox{产地}{运费}{销地}& \BLOCK{for idx in range(1, dp.demand|length + 1)}$B_{\VAR{idx}}$ & \BLOCK{endfor} 产量\\ \hline
 \BLOCK{ for s_idx in range(2) }
 $A_{\VAR{s_idx + 1}}$& \BLOCK{ for d_idx in range(dp.demand|length) }$\VAR{cost[s_idx][d_idx][0]|int}+\VAR{cost[s_idx][d_idx][1]|int}* x_{\VAR{s_idx+1}\VAR{d_idx+1}}$ & \BLOCK{endfor} \VAR{dp.supply[s_idx]}\\ \hline
 \BLOCK{ endfor }
 销量& \VAR{dp.demand|join("&")}& \\ \hline\hline
\end{mytabular}
\end{table}
 
</p>
求一个调运方案,使总费用最小{(约定运输量必须为整数)}。

\BLOCK{if after_question_description}\VAR{after_question_description}\BLOCK{endif}
\BLOCK{endif} \#{show_question}


\BLOCK{if show_answer_explanation}
<hr>
<hr>
详细的建模和求解过程:
<hr>
#### 1、动态规划建模




(1) 阶段数:$n=\VAR{dp.demand|length}$,第$k$阶段是$A_1$与$B_k$的运输量决策。

(2) 状态变量:$s_k$表示第$k$阶段初产地$A_1$**尚未运出的量**,有$s_1=\VAR{dp.total_resource}$。

(3) 决策变量:$x_{1k}$表第$k$阶段产地$A_1$运往$B_k$的量。

(4) 状态转移:$s_{k+1}=s_k-x_{1k}$.

(5) 动态规划基本方程:
<p align="middle">

\large
\VAR{dp.get_formula_tex()}

</p>
其中,阶段指标函数$g_{1k}+g_{2k}$为第$k$阶段的运输费用。

此外,模型还需满足运输量平衡条件:
\begin{align*}
\left\{\begin{array}{l}
\BLOCK{for idx in range(dp.n_stages)}
x_{2\VAR{idx+1}}=\VAR{dp.demand[idx]}-x_{1\VAR{idx+1}} \BLOCK{if not loop.last} \\ \BLOCK{endif}
\BLOCK{endfor}
\end{array}\right.
\end{align*}

</br>
<hr>
\BLOCK{endif}\#{show_answer_explanation}
