        
某供应商有A、B、C三个供货地,向某制造商供应零件,该制造商有I、II、III三个工厂,各工厂的需求量、各供货地的产量(单位:千件),以及单位运价(千元/千件)如下表所示。试求使总运费最少的调运方案。
    
        
    

    <p align="middle">
    
    \begin{table}[H]\centering \topcaption{}
        \begin{mytabular}{c|ccc|c}\hline\hline \diagbox{产地}{单位运费}{销地}
        & I&II&III& 供应量\\ \hline
          A  & 10&15&5&5 \\
          B  & 4&6&9&10 \\
          C  & 8&7&11&15 \\
         \hline
        需求量&6&9&10 & \\
        \hline\hline
        \end{mytabular}\end{table}
        
\begin{keytable}[H]\centering \topcaption{}
        \begin{mytabular}{c|cccc|c}\hline\hline \diagbox{产地}{单位运费}{销地}
        & I&II&III&IV& 供应量\\ \hline
          A  & 10&15&5&0&5 \\
          B  & 4&6&9&0&10 \\
          C  & 8&7&11&0&15 \\
         \hline
        需求量&6&9&10&5 & 30\\
        \hline\hline
        \end{mytabular}\end{keytable}
        
                \def\ori{  {  {8,7,11,0},{4,6,9,0},{10,15,5,0}  }  }
        \def\sremainder{
                \draw (sr0) node{\footnotesize 15};
                \draw (sr1) node{\footnotesize 10};
                \draw (sr2) node{\footnotesize 5};
                }
        \def\dremainder{
                \draw (dr0) node{\footnotesize 6};
                \draw (dr1) node{\footnotesize 9};
                \draw (dr2) node{\footnotesize 10};
                \draw (dr3) node{\footnotesize 5};
                }
        \def\totalx{\footnotesize 30}

        \noindent\begin{minipage}[c]{0.5\textwidth}
        \parindent 2em

        \begin{keytable}[H]
        \centering
        \topcaption{}
        \def\flows{  {  {"","9","1","5",""},{"6","","4","",""},{"","","5","",""}  }  }
        \begin{tikzpicture}[>=stealth]\draw[thick, draw=black] (0,0) grid (4,3);%%%无运输图
        \draworiginlabelx{\ori}{2}{3}
        \drawflow{\flows}{2}{3}
        \sremainder
        \dremainder
                \node at (B0) {I};
                \node at (B1) {II};
                \node at (B2) {III};
                \node at (B3) {IV};
                \node at (A2) {A};
                \node at (A1) {B};
                \node at (A0) {C};
        \end{tikzpicture}
        \end{keytable}

        \end{minipage}%
        \hskip 2em
                \noindent\begin{minipage}[c]{0.5\textwidth}
        \parindent 2em
        检验数表为:

        \begin{keytable}[H]
        \centering \topcaption{检验数表}%\label{tab:jysb1}
        \begin{mytabular}{c||C{2.5em}|C{2.5em}|C{2.5em}|C{2.5em}  }
        \hline\hline
        & I & II & III & IV  \\ \hline\hline

                A & 10&14&&6\\ \hline
                B & &1&&2\\ \hline
                C & 2&&&\\ \hline
                \hline
        \end{mytabular}
        \end{keytable}

        \end{minipage}

        
        
                
    </p>