\BLOCK{- if show_question -}
求解下图中从$\VAR{source}$至$\VAR{target}$的最短路径及最短距离.


\usepackage{tikz}
\usetikzlibrary{graphs,graphs.standard,graphdrawing,quotes,shapes,arrows.meta}
\usegdlibrary{force}


<p align="middle">


\VAR{g.as_latex()}


</p>

(注:若有多条最短路,应全部指出\BLOCK{if g.has_6_9_edge_weight};无法区分数字6或9时,弧上数字与边的关系是数字在上,边在下\BLOCK{endif})

\BLOCK{ endif } \#{ show_question }

\BLOCK{- if show_answer -}



\BLOCK{ if show_dijkstra }

\BLOCK{ if not dijkstra_is_allowed }
<strong>解:</strong>图中存在权重为负的弧,Dijkstra解法不适用.
\BLOCK{ else }

\BLOCK{ if dijkstra_result }

<strong>解:</strong>用简化的Dijkstra标号法求解如下:

<p align="middle">答案表\VAR{answer_table_iters.next()}</strong></br>

\begin{keytable}[H]
\centering \xiaosi
\begin{tabular}{r\BLOCK{ for i in range(g.graph|length -1) }C{1cm}\BLOCK{ endfor }C{1.2cm}l}
\multirow{ \VAR{dijkstra_result.n_pred_lines} }*{$pred_j$}

\BLOCK{ for pred_line in dijkstra_result.tex_pred_list }& \BLOCK{ for i in range(g.graph|length) }
\VAR{pred_line[i]} & \BLOCK{ endfor } \\
\BLOCK{ endfor }
\cline{2-\VAR{g.graph|length+1}}
&(\BLOCK{ for node in g.node_label_dict } $\VAR{g.node_label_dict[node]}$ \BLOCK{ if not loop.last}&\BLOCK{endif} \BLOCK{ endfor })~~&\\
\BLOCK{ for line in dijkstra_result.tex_L_list }\BLOCK{ set outer_loop = loop }
$L_{\VAR{ loop.index0 }}=$ &( \BLOCK{ for node in line } \VAR{line[node]} \BLOCK{ if not loop.last}&\BLOCK{endif} \BLOCK{ endfor })\BLOCK{if not outer_loop.last};&\BLOCK{else}.&\BLOCK{endif}\\
\BLOCK{ endfor }
\end{tabular}
\end{keytable}

</p>

最短路长为\VAR{(g.final_dist[-1][g.graph|length -1])|int},\BLOCK{ if dijkstra_result.shortest_path_list|length > 1}有\VAR{dijkstra_result.shortest_path_list|length}条最短路径:\BLOCK{ else } 最短路径为:\BLOCK{endif}

$$\begin{align*}
\BLOCK{ for path in dijkstra_result.shortest_path_tex_list }
&\VAR{path|join("\\rightarrow ")} \BLOCK{ if not loop.last }; \\
\BLOCK{else}.
\BLOCK{endif}
\BLOCK{ endfor }
\end{align*}$$

\BLOCK{ endif } \#{ dijkstra_result }
\BLOCK{ endif } \#{ dijkstra_is_allowed }
\BLOCK{ endif } \#{ show_dijkstra }


\BLOCK{ if show_bellman_ford }

\BLOCK{ if bellman_ford_result }
\BLOCK{ if dijkstra_result }<hr>\BLOCK{endif}

\BLOCK{ if not show_dijkstra }<strong>解:</strong>\BLOCK{endif}用Bellman-Ford解法求解如下:

<p align="middle"><strong>答案表\VAR{answer_table_iters.next()}</strong></br>

\begin{keytable}[H]
 \centering
 \begin{mytabular}{\BLOCK{for i in range(g.graph|length + 1)}c|\BLOCK{endfor}\BLOCK{for i in range(bellman_ford_result.nit + 1)}|c\BLOCK{endfor}}\hline
 \hline
 $w_{ij}$ & \BLOCK{ for node in g.node_label_dict } $\VAR{g.node_label_dict[node]}$ & \BLOCK{endfor} \BLOCK{for i in range(1, bellman_ford_result.nit + 1)}$d_j^{(\VAR{i})}$ & \BLOCK{endfor} $pred_j$ \\
 \hline\hline
 \BLOCK{for node in g.node_label_dict}
 $\VAR{g.node_label_dict[node]}$ & \VAR{bellman_ford_result.matrix[node]|join("&")} & \VAR{bellman_ford_result.dist_list_by_node[node]|join("&")} & \VAR{bellman_ford_result.final_pred_tex_dict[node]} \\
 \hline
 \BLOCK{endfor}
 \hline \end{mytabular}%
 \end{keytable}


</p>

最短路长为\VAR{(g.final_dist[-1][g.graph|length -1])|int},\BLOCK{ if bellman_ford_result.shortest_path_list|length > 1}有\VAR{bellman_ford_result.shortest_path_list|length}条最短路径:\BLOCK{ else } 最短路径为:\BLOCK{endif}

$$\begin{align*}
\BLOCK{ for path in bellman_ford_result.shortest_path_tex_list }
&\VAR{path|join("\\rightarrow ")} \BLOCK{ if not loop.last }; \\
\BLOCK{else}.
\BLOCK{endif}
\BLOCK{ endfor }
\end{align*}$$

\BLOCK{ endif } \#{ bellman_ford_result }
\BLOCK{ endif } \#{ show_bellman_ford }

\BLOCK{ endif } \#{ show_answer }