\BLOCK{- if show_question -}
求解下图中从$\VAR{source}$至$\VAR{target}$的最短路径及最短距离(若有多条最短路,请全部指出).


\usepackage{tikz}
\usetikzlibrary{graphs,graphs.standard,graphdrawing,quotes,shapes,arrows.meta}
\usegdlibrary{force}


<p align="middle">


\VAR{g.as_latex()}


</p>

\BLOCK{ endif } \#{ show_question }

\BLOCK{- if show_answer -}



\BLOCK{ if show_dijkstra }

\BLOCK{ if not dijkstra_result }
图中存在权重为负的弧,Dijkstra解法不适用.

\BLOCK{ else }

用简化的Dijkstra标号法求解如下:

<p align="middle">


\begin{center}\xiaosi
\begin{tabular}{r\BLOCK{ for i in range(g.graph|length -1) }C{1cm}\BLOCK{ endfor }C{1.2cm}l}
\multirow{ \VAR{dijkstra_result.n_pred_lines} }*{$pred_j$}

\BLOCK{ for pred_line in dijkstra_result.tex_pred_list }& \BLOCK{ for i in range(g.graph|length) }
\VAR{pred_line[i]} & \BLOCK{ endfor } \\
\BLOCK{ endfor }
\cline{2-\VAR{g.graph|length+1}}
&(\BLOCK{ for node in g.node_label_dict } $\VAR{g.node_label_dict[node]}$ \BLOCK{ if not loop.last}&\BLOCK{endif} \BLOCK{ endfor })~~&\\
\BLOCK{ for line in dijkstra_result.tex_L_list }\BLOCK{ set outer_loop = loop }
$L_{\VAR{ loop.index0 }}=$ &( \BLOCK{ for node in line } \VAR{line[node]} \BLOCK{ if not loop.last}&\BLOCK{endif} \BLOCK{ endfor })\BLOCK{if not outer_loop.last};&\BLOCK{else}.&\BLOCK{endif}\\
\BLOCK{ endfor }
\end{tabular}
\end{center}


</p>

最短路长为\VAR{(g.final_dist[g.graph|length -1])|int}, 有\VAR{dijkstra_result.shortest_path_list|length}条最短路径:
$$\begin{align*}
\BLOCK{ for path in dijkstra_result.shortest_path_tex_list }
&\VAR{path|join("\\rightarrow ")} \BLOCK{ if not loop.last }; \\ \BLOCK{else}.
\BLOCK{endif}
\BLOCK{ endfor }
\end{align*}$$

\BLOCK{ endif } \#{ dijkstra_result }


\BLOCK{ endif } \#{ show_dijkstra }

\BLOCK{ endif } \#{ show_answer }

