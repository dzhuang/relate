\BLOCK{- if show_question -}
\BLOCK{- if pre_description -}
\VAR{pre_description}
\BLOCK{- endif -} \#{ pre_description }



<p align="middle">

\begin{keytable}[H]\centering
\begin{tabular}{c|c|c|\BLOCK{ for x in range(0, lp.solutionPhase2.all_variable_str_list|length) }
C{1cm}\BLOCK{ endfor }|c}\hline\hline
& \multicolumn{2}{c|}{$c_j$}\BLOCK{ for c in lp.solutionPhase2.goal_str_list }&\VAR{c}\BLOCK{ endfor }&\multirow{2}*{$\mathbf{b}$} \\ \cline{2-\VAR{lp.solutionPhase2.all_variable_str_list|length + 3 }}
& $\mathbf{C_B}$&$\mathbf{X_B}$\BLOCK{ for x in lp.solutionPhase2.all_variable_str_list }&\VAR{x}\BLOCK{ endfor }&\\ \hline
\BLOCK{ for i in range(0, lp.solutionPhase2.tableau_str_list|length) }
\BLOCK{ if loop.first or loop.last }
\multirow{\VAR{lp.solutionPhase2.tableau_str_list[0]|length}}{*}{\begin{sideways} \BLOCK{ if loop.first}初始表\BLOCK{ else }最优表\BLOCK{ endif } \end{sideways}}
\BLOCK{ for j in range(0, lp.solutionPhase2.cb_str_list[i]|length + 1) }
\BLOCK{ if not loop.last -} & \VAR{lp.solutionPhase2.cb_str_list[i][j]}&\VAR{lp.solutionPhase2.xb_str_list[i][j]}&
\BLOCK{- else -}
& \multicolumn{2}{c|}{$\bar c_j$}&
\BLOCK{- endif } \VAR{(lp.solutionPhase2.tableau_str_list[i][j]|join("&"))|replace("[$\mathbf{", "$")|replace("}$]","$") } \\

\BLOCK{- if loop.revindex0 == 1 } \cline{2-\VAR{lp.solutionPhase2.all_variable_str_list|length + 4 }} \BLOCK{ endif }
\BLOCK{- if loop.revindex0 == 0 } \hline\hline \BLOCK{ endif }

\BLOCK{ endfor }
\BLOCK{ endif }
\BLOCK{ endfor }
\hline\end{tabular}
\end{keytable}%

</p>


\BLOCK{ if final_description -}
\VAR{final_description}
\BLOCK{ endif }



\BLOCK{ if after_description -}
\VAR{after_description}
\BLOCK{ endif -} \#{ after_description }


\BLOCK{- endif -} \#{ show_question }

\BLOCK{ if show_blank -}
该问题的对偶问题的<strong>最优解</strong>是[[blank1]]
\BLOCK{ endif -} \#{ after_description }

\BLOCK{ if show_blank_answer }
blank1:
    type: ShortAnswer
    width: 10em
    hint: <p><strong>输入格式示例:</strong></p><ul><li>(1,2,3,4)</li><li>(1,2,3,4,5)^T</li></ul><p><strong>说明:</strong><ol><li>必须使用英文输入法输入;<li>输入必须是解向量的形式,即用圆括号包围;</li><li>如果需要转置,需要在末尾加上<strong>^T</strong>表示转置.</li></ol>
    correct_answer:
    - type: float_list_with_wrapper
      forced_left_wrapper: ["("]
      forced_right_wrapper: [")", ")^T"]
      rtol: 0.0001
      atol: 0.0001
      value: "\VAR{answer1}"
    - type: float_list_with_wrapper
      forced_left_wrapper: ["("]
      forced_right_wrapper: [")", ")^T"]
      rtol: 0.0001
      atol: 0.0001
      value: "\VAR{answer2}"
\BLOCK{ endif -} \#{ after_description }