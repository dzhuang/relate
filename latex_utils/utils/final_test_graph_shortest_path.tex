

求解下图中从$\VAR{source}$至$\VAR{target}$的最短路径及最短距离.

\BLOCK{if show_id}
%------------------------------------------------------\VAR{id}---------------------------------------------------
\BLOCK{endif}

\iffalse
% ---------------------------------以下部分需独立编译为shortestpath\VAR{id}.pdf--------------------------------------

\RequirePackage{luatex85}
\VAR{g.as_latex()}

% ---------------------------------以上部分需独立编译shortestpath\VAR{id}.pdf--------------------------------------
\fi

\begin{center}
\includegraphics[scale=1]{shortestpath\VAR{id}.pdf}
\end{center}


\BLOCK{if show_answer -}
\key{\daan{\todos[inline]{必须有明确的算法,观察法即使答案正确也不得分!}

\BLOCK{ if show_dijkstra }

\BLOCK{ if not dijkstra_is_allowed }
\jie 图中存在权重为负的弧,Dijkstra解法不适用.
\BLOCK{ else }

\BLOCK{ if dijkstra_result }

\jie 用简化的Dijkstra标号法求解如下:

\begin{table}[!h]
\centering
\tabtopcaption{}
\begin{tabular}{r\BLOCK{ for i in range(g.graph|length -1) }C{1cm}\BLOCK{ endfor }C{1.2cm}l}
\multirow{ \VAR{dijkstra_result.n_pred_lines} }*{$pred_j$}

\BLOCK{ for pred_line in dijkstra_result.tex_pred_list }& \BLOCK{ for i in range(g.graph|length) }
\VAR{pred_line[i]} & \BLOCK{ endfor }\BLOCK{if loop.first}\todos{各节点的前趋节点不能写错,写错或少写少则扣1分,此处最多扣5分}\BLOCK{endif}\\
\BLOCK{ endfor }
\cline{2-\VAR{g.graph|length+1}}
&(\BLOCK{ for node in g.node_label_dict } $\VAR{g.node_label_dict[node]}$ \BLOCK{ if not loop.last}&\BLOCK{endif} \BLOCK{ endfor })~~&\\
\BLOCK{ for line in dijkstra_result.tex_L_list }\BLOCK{ set outer_loop = loop }
$L_{\VAR{ loop.index0 }}=$ &( \BLOCK{ for node in line } \VAR{line[node]} \BLOCK{ if not loop.last}&\BLOCK{endif} \BLOCK{ endfor })\BLOCK{if not outer_loop.last};&\BLOCK{else}.&\BLOCK{endif}\\
\BLOCK{ endfor }
\todosv{行数应与节点数相同,多/少一行扣2分。}{-3cm}
\end{tabular}
\end{table}

最短路长为\VAR{g.final_dist[g.n_node-1]|int}, \BLOCK{ if dijkstra_result.shortest_path_list|length > 1}有\VAR{dijkstra_result.shortest_path_list|length}条最短路径:\BLOCK{ else } 最短路径为:\BLOCK{endif}

\begin{align*}
\BLOCK{ for path in dijkstra_result.shortest_path_tex_list }
&\VAR{path|join("\\rightarrow ")} \BLOCK{ if not loop.last }; \\
\BLOCK{else}.
\BLOCK{endif}
\BLOCK{ endfor }
\end{align*}

\todos[inline]{注:1、必须写出最短路径和最短距离,前者未写扣2分,后者未写扣1分;\\ 2、有多条最短路径,少1条扣2分;\\ 3、最短路长应直接写出,而不是做加法!否则扣1分。}

\BLOCK{ endif } \#{ dijkstra_result }
\BLOCK{ endif } \#{ dijkstra_is_allowed }
\BLOCK{ endif } \#{ show_dijkstra }


\BLOCK{ if show_bellman_ford }

\BLOCK{ if bellman_ford_result }
\BLOCK{ if dijkstra_result }\BLOCK{endif}

\BLOCK{ if not show_dijkstra }\jie \BLOCK{endif}
\BLOCK{ if not dijkstra_is_allowed }
\todos[inline]{必须采用Bellman-Ford算法,dijkstra算法不得分;观察法和即使答案正确也不得分!}
\BLOCK{endif}
用Bellman-Ford解法求解如下:

\begin{table}[!h]
 \centering
 \tabtopcaption{}
 \begin{mytabular}{\BLOCK{for i in range(g.graph|length + 1)}c|\BLOCK{endfor}\BLOCK{for i in range(bellman_ford_result.nit + 1)}|c\BLOCK{endfor}}\hline
 \hline
 $w_{ij}$ & \BLOCK{ for node in g.node_label_dict } $\VAR{g.node_label_dict[node]}$ & \BLOCK{endfor} \BLOCK{for i in range(1, bellman_ford_result.nit + 1)}$d_j^{(\VAR{i})}$ & \BLOCK{endfor} $pred_j$ \\
 \hline\hline
 \BLOCK{for node in g.node_label_dict}
 $\VAR{g.node_label_dict[node]}$ & \VAR{bellman_ford_result.matrix[node]|join("&")} & \VAR{bellman_ford_result.dist_list_by_node[node]|join("&")} & \VAR{bellman_ford_result.final_pred_tex_dict[node]} \\
 \hline
 \BLOCK{endfor}
 \hline \end{mytabular}%
 \end{table}

\todos[inline]{答案必须与此表完全一致. 少一列扣2分;有多个前趋节点的,少写一个扣1分}
最短路长为\VAR{g.final_dist[g.n_node-1]|int},\BLOCK{ if bellman_ford_result.shortest_path_list|length > 1}有\VAR{bellman_ford_result.shortest_path_list|length}条最短路径:\BLOCK{ else } 最短路径为:\BLOCK{endif}

\begin{align*}
\BLOCK{ for path in bellman_ford_result.shortest_path_tex_list }
&\VAR{path|join("\\rightarrow ")} \BLOCK{ if not loop.last }; \\
\BLOCK{else}.
\BLOCK{endif}
\BLOCK{ endfor }
\end{align*}
\todos[inline]{注:1、必须写出最短路径和最短距离,前者未写扣2分,后者未写扣1分;\\ 2、有多条最短路径,少1条扣2分;\\ 3、最短路长应直接写出,而不是做加法!否则扣1分。}
\BLOCK{ endif } \#{ bellman_ford_result }
\BLOCK{ endif } \#{ show_bellman_ford }

}} \#{key and daan}

\BLOCK{ endif } \#{ show_answer }