\BLOCK{- if show_question -}
\BLOCK{- if problem_description_pre -}
\VAR{problem_description_pre}
\BLOCK{- endif -}求解下图中从$\VAR{source}$至$\VAR{target}$的\BLOCK{- if not show_max_flow_network -}最大流及\BLOCK{- endif -}最小割,并思考最小**割集**是什么。图中弧上的数字表示其容量\BLOCK{- if show_max_flow_network -}及取得某个最大流时的流量\BLOCK{- endif -}.


\usepackage{tikz}
\usetikzlibrary{graphs,graphs.standard,graphdrawing,quotes,shapes,arrows.meta}
\usegdlibrary{force}


\BLOCK{- if show_flow_network_capacity_only -}
<p align="middle">

\RequirePackage{luatex85}
\VAR{g.as_latex(node_distance=node_distance, hidden_node_list=hidden_node_list, regenerate_node_label_dict=True)}


</p>
\BLOCK{- endif -}

\BLOCK{- if show_max_flow_network -}

<p align="middle">

\RequirePackage{luatex85}
\VAR{g.get_max_flow_latex(node_distance="3cm", hidden_node_list=hidden_node_list, regenerate_node_label_dict=True)}


</p>
\BLOCK{- endif -}


\BLOCK{if g.has_6_9_edge_weight}(注:无法区分数字6或9时,弧上数字与边的关系是数字在上,边在下)\BLOCK{endif}

\BLOCK{ endif } \#{ show_question }

\BLOCK{ if show_blank -}
\BLOCK{ if blank1_desc -}\VAR{blank1_desc}\BLOCK{ endif -}[[blank1]]<span class="hidden-xs hidden-sm">,</span>\BLOCK{ if blank2_desc -}\VAR{blank2_desc}\BLOCK{ endif -}[[blank2]]<span class="hidden-sm">.</span>
\BLOCK{ endif -} \#{ after_description }

\BLOCK{ if show_blank_answer }
blank1:
    type: ShortAnswer
    width: 5em
    weight: 0.5
    correct_answer:
    - type: float
      rtol: 0.001
      atol: 0.01
      value: \VAR{max_flow_value}

blank2:
    type: ShortAnswer
    width: 15em
    prepended_text: "{"
    appended_text: "}"
    weight: 0.5
    hint_title: 输入格式示例
    hint: <p><strong>仅输入节点的下标</strong>,例如,如果$S=\{v_1, v_2, v_3, v_4\}$,则输入</p><ul><li>1, 2, 3, 4</li><li>或2,1,3,4</li></ul>
    correct_answer:
    - type: float_list_with_wrapper
      as_set: True
      set_allowed_range: \VAR{set_allowed_range}
      rtol: 0.01
      atol: 0.01
      value: "\VAR{min_cut_set_s}"


\BLOCK{ endif -}


\BLOCK{if show_answer_explanation}

- 最小割中不应包含将问题化为等价单源点单汇点时引入的节点和弧,所以虚拟的弧的容量应设为无穷大

- 要正确地找出最小割,必须在找出最大流后,对所有可能标号的点标上号,并将这些节点作为最小割集中的$S$集合,容易出错的地方是遗漏了一些非零流后向弧上的节点.

- 本问题的最小**割集**为:\VAR{cut_set_str}.

\BLOCK{ endif -}