
\BLOCK{if show_id}
%------------------------------------------------------\VAR{id}---------------------------------------------------
\BLOCK{endif}
\\VAR{ajbj}{下图为从$\VAR{source}$至$\VAR{target}$的一个最大流,图中弧上的数字先后表示其容量及取得某个最大流时的流量.

\BLOCK{if show_flow_network_capacity_only}
\RequirePackage{luatex85}
\VAR{g.as_latex(node_distance=node_distance)}
\BLOCK{- endif -}

\iffalse
% ---------------------------------以下部分需独立编译为maxflow\VAR{id}.pdf--------------------------------------

\BLOCK{if show_max_flow_network}
\RequirePackage{luatex85}
\VAR{g.get_max_flow_latex(node_distance="3cm")}
\BLOCK{- endif}

% ---------------------------------以上部分需独立编译maxflow\VAR{id}.pdf--------------------------------------
\fi

\begin{center}
\includegraphics[width=0.5\textwidth]{maxflow\VAR{id}.pdf}
\end{center}

\BLOCK{if g.has_6_9_edge_weight}(注:无法区分数字6或9时,弧上数字与边的关系是数字在上,边在下)\BLOCK{endif}

\noindent 则其最小割{\hei 割集}为:\{\blank{\longb}{$\VAR{cut_set_string}$}\todos{次序可随意,多、少都不得分.}\}.
}%
