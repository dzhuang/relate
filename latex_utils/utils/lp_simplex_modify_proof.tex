\BLOCK{- if show_question -}
证明以下线性规划问题
\BLOCK{ if lp }
$$\begin{alignat*}{20}
\VAR{lp.get_goal_str()}\\
\text{s.t.}\quad
\BLOCK{ for c in lp.constraints_str_list }
\VAR{c}
\BLOCK{ endfor }
\VAR{lp.sign_str}
\end{alignat*}$$
\BLOCK{ endif }
存在一个以$(\BLOCK{ for idx in opt_basis}x_{\VAR{idx+1}}\BLOCK{if not loop.last}, \BLOCK{ endif }\BLOCK{ endfor })^T$为基变量向量的最优解.

\BLOCK{- endif -} \#{ show_question }

\BLOCK{- if show_answer}
\BLOCK{ set answer_table_counter=0 }

\BLOCK{ if simplex_pre_description -}
\VAR{simplex_pre_description}
\BLOCK{ endif } \#{ simplex_pre_description }

\BLOCK{ if lp }
<strong>证明:</strong> \BLOCK{ if lp.solutionCommon.existing_basic_variable_str_list }
现有变量$\VAR{lp.solutionCommon.existing_basic_variable_str_list|join(", ")}$
可直接作为基变量使用,\BLOCK{ endif -}
\BLOCK{ if lp.solutionCommon.slack_str_list_intro or lp.solutionCommon.neg_slack_str_list_intro }引入\BLOCK{ endif -}
\BLOCK{- if lp.solutionCommon.slack_str_list_intro }松弛变量$\VAR{lp.solutionCommon.slack_str_list_intro|join(",\,") }$\BLOCK{ endif -}
\BLOCK{- if lp.solutionCommon.slack_str_list_intro and lp.solutionCommon.neg_slack_str_list_intro},\BLOCK{ endif -}
\BLOCK{- if lp.solutionCommon.neg_slack_str_list_intro }剩余变量$\VAR{lp.solutionCommon.neg_slack_str_list_intro|join(",\,") }$\BLOCK{ endif -}
\BLOCK{ if lp.solutionCommon.slack_str_list_intro or lp.solutionCommon.neg_slack_str_list_intro },将问题标准化. \BLOCK{ endif }

\BLOCK{ endif }

\BLOCK{ if lp.solutionCommon.method == "modified_simplex" }

\BLOCK{ if standardized_lp }
$$\begin{alignat*}{20}
\VAR{standardized_lp.get_goal_str()}\\
\text{s.t.}\quad
\BLOCK{ for c in standardized_lp.constraints_str_list }
\VAR{c}
\BLOCK{ endfor }
\VAR{standardized_lp.sign_str}
\end{alignat*}$$
\BLOCK{ endif }


%# 改进单纯形法开始

\BLOCK{ for idx in range(lp.solutionPhase2.modi_basis_list|length) }
\BLOCK{ if loop.last }
<strong>(1). 证明基的最优性</strong> 当基变量向量为$\mathbf{X_B}=(\VAR{lp.solutionPhase2.modi_basis_list[idx]|join(", ")})^T$时,有:
$$\mathbf{B}=(\VAR{lp.solutionPhase2.modi_bp_list[idx]|join(", ")})=\begin{bmatrix}
\VAR{lp.solutionPhase2.modi_B_list[idx]|join("")}
\end{bmatrix}\quad \Rightarrow \quad \mathbf{B}^{-1}=\begin{bmatrix}
\VAR{lp.solutionPhase2.modi_B_1_list[idx]|join("")}
\end{bmatrix}$$

计算非基变量的检验数$\bar c_j$:
$$\begin{align*}
\BLOCK{ for cjbar_idx in lp.solutionPhase2.modi_CJBAR_idx_list[idx] }
\bar c_{\VAR{cjbar_idx+1}}&=c_{\VAR{cjbar_idx+1}}-\mathbf{C_B}\mathbf{B}^{-1}\mathbf{p}_{\VAR{cjbar_idx+1}}=\VAR{lp.solutionPhase2.modi_CJBAR_list[idx][cjbar_idx]} \\
\BLOCK{ endfor }
\end{align*}$$



\VAR{lp.solve_status_reason},该基变量向量满足最优化条件.

<strong>(2). 证明当前的解为基本可行解</strong>:

\BLOCK{ if lp.solve_status<1 }
$$\mathbf{X_B}= \mathbf{\bar b}=\mathbf{B}^{-1}\mathbf{b}=\begin{bmatrix} \VAR{lp.solutionPhase2.modi_b_list[idx]|join(" \\\\ ")} \end{bmatrix}$$
\BLOCK{ endif -}
所有变量取值非负,所以该基变量向量对应的基本解<strong>可行</strong>.

<strong>证毕</strong>.

说明:检验数判定只能判定该基变量向量对应的是一个满足了优化条件的解,但不能保证一定是满足了非负约束的解.

\BLOCK{ endif -}\#{ loop.last }

\BLOCK{ endfor -}

\BLOCK{ endif -}

\BLOCK{- if simplex_after_description -}
\VAR{simplex_after_description}
\BLOCK{ endif -}

\BLOCK{ endif -} \#{ show_answer }