\BLOCK{- if show_question -}
W公司市场部门计划在\VAR{dp.project_list|join("、")}三地共租用\VAR{dp.total_resource}台宣传车开展其新产品的现场推介及促销活动。根据往年的宣传效果统计数据,在不同城市、租用不同数量宣传车时的带来的促销收益的估计值(单位:万元)如下表所示,表中被划去的单元格表示不允许相应的方案:


<p align="middle"></br>

\begin{table}[H]
 \centering
 \topcaption{各地宣传车辆的配备数量与收益的关系表}
 \begin{mytabular}{c|\BLOCK{for d in dp.project_list}c\BLOCK{if not loop.last}|\BLOCK{endif}\BLOCK{endfor}}
 \hline\hline
\diagbox{车辆数}{收益$g_k$}{城市}&\VAR{dp.project_list|join("&")}\\ \hline
\BLOCK{for idx in range(dp.decision_set|length)}
\VAR{dp.decision_set[idx]}&\VAR{dp.gain_str_list_transposed[idx]|join("&")}\\ \hline
\BLOCK{endfor}
 \hline
 \end{mytabular}
 \end{table}
 
</p>

问:各城市各应租用多少台宣传车可使该公司所获利的促销收益最大?


\BLOCK{endif} \#{show_question}

\BLOCK{if after_question_description}\VAR{after_question_description}\BLOCK{endif}

\BLOCK{if show_answer_explanation}
<hr>
<hr>
详细的建模和求解过程:
<hr>
#### 1、动态规划建模




(1) 阶段数$n=\VAR{dp.n_stages}$,以\VAR{dp.project_list|join("、")}作为城市编号的次序,以向第$k$个城市分配车辆的决策作为第$k$阶段。$k=\VAR{range(1, dp.n_stages+1)|join(",")}$;

(2) 状态变量$s_k$为第$k$阶段初**尚未分配的车辆数**,则$s_1=\VAR{dp.total_resource}$(亦即$S_1=\{\VAR{dp.total_resource}\}$);

(3) 决策变量$x_k$为第$k$阶段向第$k$个城市分配的车辆数,$k=\VAR{range(1, dp.n_stages+1)|join(",")}$;

(4) 状态转移方程为:
$$
\begin{align*}
 &s_{k+1}=s_k-x_k, ~~k=\VAR{range(1, dp.n_stages+1)|join(",")}
\end{align*}
$$其中$s_1=\VAR{dp.total_resource}$;

(5) 动态规划基本方程为:
<p align="middle">

\large
\VAR{dp.get_formula_tex()}

</p>
其中,阶段指标$g_k(s_k, x_k)$为第$k$阶段向第$k$个城市分配$x_k$辆车带来的收益。
</br>
<hr>
\BLOCK{endif}\#{show_answer_explanation}
