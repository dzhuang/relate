
设有产量分别为30,40,30的三个原料产地,欲将原材料运往销量分别为25,20,40的三个销地,单位运价如下表所示。试求总运费最省的调运方案。
    



<p align="middle">

\begin{table}[H]\centering \topcaption{}
\begin{mytabular}{c|ccc|c}\hline\hline \diagbox{产地}{单位运费}{销地}
&  B1 &  B2 &  B3 &  供应量\\ \hline
   A1  & 5&6&9&30 \\
   A2  & 9&4&8&40 \\
   A3  & 10&7&5&30 \\
 \hline
需求量&25&20&40 & \\
\hline\hline
\end{mytabular}\end{table}

\begin{keytable}[H]\centering \topcaption{}
\begin{mytabular}{c|cccc|c}\hline\hline \diagbox{产地}{单位运费}{销地}
&  B1 &  B2 &  B3 &  B4 &  供应量\\ \hline
   A1  & 5&6&9&0&30 \\
   A2  & 9&4&8&0&40 \\
   A3  & 10&7&5&0&30 \\
 \hline
需求量&25&20&40&15 & 100\\
\hline\hline
\end{mytabular}\end{keytable}

\def\ori{  {  {10,7,5,0},{9,4,8,0},{5,6,9,0}  }  }
\def\sremainder{
\draw (sr0) node{\footnotesize 30};
\draw (sr1) node{\footnotesize 40};
\draw (sr2) node{\footnotesize 30};
}
\def\dremainder{
\draw (dr0) node{\footnotesize 25};
\draw (dr1) node{\footnotesize 20};
\draw (dr2) node{\footnotesize 40};
\draw (dr3) node{\footnotesize 15};
}
\def\total{\footnotesize 100}

\noindent\begin{minipage}[c]{0.5\textwidth}
\parindent 2em

\begin{keytable}[H]
\centering
\topcaption{}
\def\flows{  {  {"","","30","",""},{"","20","10","10",""},{"25","","","5",""}  }  }
\begin{tikzpicture}[>=stealth]\draw[thick, draw=black] (0,0) grid (4,3);%%%无运输图
\draworiginlabel{\ori}{2}{3}
\drawflow{\flows}{2}{3}
\sremainder
\dremainder
\end{tikzpicture}
\end{keytable}

\end{minipage}%
\hskip 2em
\noindent\begin{minipage}[c]{0.5\textwidth}
\parindent 2em
检验数表为:

\begin{keytable}[H]
\centering \topcaption{检验数表}%\label{tab:jysb1}
\begin{mytabular}{c||C{2.5em}|C{2.5em}|C{2.5em}|C{2.5em}  }
\hline\hline
& $B1$ & $B2$ & $B3$ & $B4$  \\ \hline\hline

$A1$ & &2&1&\\ \hline
$A2$ & 4&&&\\ \hline
$A3$ & 8&6&&3\\ \hline
\hline
\end{mytabular}
\end{keytable}

\end{minipage}




</p>